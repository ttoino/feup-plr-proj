\documentclass[conference]{IEEEtran}

%\IEEEoverridecommandlockouts
% The preceding line is only needed to identify funding in the first footnote. If that is unneeded, please comment it out.

\usepackage{cite}
\usepackage{amsmath,amssymb,amsfonts}
\usepackage{algorithmic}
\usepackage{graphicx}
\usepackage{textcomp}
\usepackage{xcolor}
\usepackage{pbalance} % for balancing the last page
\usepackage{enumitem} % for enumerated lists that can continue numbering
\usepackage{soul} % for underlining text with wrapping
\usepackage[htt]{hyphenat} % for text hyphenation and wrapping


\def\BibTeX{{\rm B\kern-.05em{\sc i\kern-.025em b}\kern-.08em
    T\kern-.1667em\lower.7ex\hbox{E}\kern-.125emX}}

\def\NOIMPL#1{\ul{#1} {\color{blue}[NOT IMPLEMENTED]\color{black}}}
\def\CONFIRM#1{\ul{#1} {\color{red}[CONFIRM]\color{black}}}

\begin{document}

\title{
    Creating Optimized Schedules for Pharmacy Workers: a CLP-based approach \\
    % {
    %     \footnotesize A CLP-based approach
    % }
}

\author{
    \IEEEauthorblockN{João Pereira}
    \IEEEauthorblockA{
        \textit{Faculdade de Engenharia} \\
        \textit{Universidade do Porto} \\
        Porto, Portugal \\
        up202007145@up.pt
    }
\and
    \IEEEauthorblockN{André Lima}
    \IEEEauthorblockA{
        \textit{Faculdade de Engenharia} \\
        \textit{Universidade do Porto} \\
        Porto, Portugal \\
        up202008169@up.pt
    }
}

\maketitle

\begin{abstract}

The scheduling of workers is a common problem in many industries.
In the pharmacy industry, scheduling is usually done by hand and heeding to the preference of individual workers becomes a complex task, even at smaller scales. Furthermore, ensuring that schedules are fair and that the workload is evenly distributed among workers is a challenging task.
This paper presents a Constraint Logic Programming (CLP) based approach to create optimized schedules for pharmacy workers. The proposed approach uses a CLP solver to find the optimal schedule for a given set of workers. The performance of the proposed approach is evaluated using a real-world dataset of pharmacy workers. The results show that the proposed approach can create optimized schedules that meet the preferences of individual workers.

\end{abstract}

\begin{IEEEkeywords}
scheduling, optimization, clp, pharmacy
\end{IEEEkeywords}

\section{Introduction}

In some industries, scheduling of workers is very easy: all workers come in at the same time and they all leave at the same time.
However, in the pharmacy industry, scheduling can become a complex task, especially for smaller pharmacies, where the number of workers is reduced and schedules are adapted to meet the preferences of the workers.

With that in mind, a pharmacy in Portugal contacted us to create a scheduling system that would replace their current manual scheduling system. The new scheduling system should be faster and create better schedules when compared to the current system.
In section \ref{section:problem-statement}, the problem and the criteria used to compare schedules will be specified. In section \ref{section:proposed-model}, the proposed approach to model the problem will be presented, with a focus on developing an implementation for SICStus Prolog 4.7.1. In section \ref{section:results}, we will evaluate the model and discuss ways to improve its performance. In section \ref{section:further-work}, ideas for further work will be presented. Finally, in section \ref{section:conclusion}, the conclusions of this paper will be presented.

\section{Problem Statement}
\label{section:problem-statement}

The problem was initially presented as a very specific set of constraints.
After analyzing the problem, some constraints were generalized so that the system can better adapt to the pharmacy's future needs.

The resulting set of constraints can be summarized as follows:

\begin{enumerate}[start]
    \item {
        \label{constraint:domain}
        The pharmacy has 13 workers and 2 locations. 12 workers can work at the first location, but only 7 workers can work at the second location.
    }
    \item {
        \label{constraint:shifts}
        The first location has 7 daytime shifts and 1 nighttime shift. The second location has 2 daytime shifts.
    }
    \item {
        \label{constraint:variable-length-shifts}
        3 out of the 7 daytime shifts in the first location have variable length. These shifts are referred to as \textbf{variable-length shifts}.
    }
    \item {
        \label{constraint:overtime-shifts}
        Some shifts in the first location have longer duration, but don't give the workers overtime pay. These shifts will be referred to as \textbf{overtime shifts}. There cannot be more than 2 of these shifts each day.
    }
    \item {
        \label{constraint:late-shifts}
        Some shifts end later than others. These shifts will be referred to as \textbf{late shits}. Each worker cannot have more than 2 of these shifts each week.
    }
    \item {
        \label{constraint:nighttime-shift-implies-daytime-shift}
        If, on a given day, a worker is assigned to a nighttime shift, then they must also be assigned to a daytime shift on the same day. That shift must end before or at 19:00.
    }
    \item {
        \label{constraint:shift-incompatibilities}
        Workers cannot be assigned to shifts that are incompatible with their daily commitments. These commitments are known ahead of time and will be referred to as a worker's \textbf{shift incompatibilities}.
    }
    \item {
        \label{constraint:weekend-schedules-known-ahead-of-time}
        Weekend schedules are known ahead of time and don't need to be calculated.
    }
    \item {
        \label{constraint:absenses-known-ahead-of-time}
        Some absenses are known ahead of time. Every day, 4 workers need to be absent.
    }
\end{enumerate}

Furthermore, certain criteria make a schedule better than others. These criteria are:

\begin{enumerate}[resume]
    \item {
        \label{criterion:worker-preferences}
        Workers should have their preferences met. For example, if a worker prefers to come to work and leave earlier, then they should be assigned to shifts that facilitate that, if possible.
    }
    \item {
        \label{criterion:rotated-shifts-distribution}
        Some shifts have a "good" or "bad" reputation and, thus, each of them needs to be equally distributed among all workers. These shifts will be referred to as \textbf{rotated shifts}.
    }
    \item {
        \label{criterion:overtime-distribution}
        Overtime should be equally distributed among all workers.
    }
    \item {
        \label{criterion:absenses-distribution}
        Absenses that are not known ahead of time should be equally distributed among all workers.
    }
    \item {
        \label{criterion:nighttime-workers-prioritzation}
        If a worker is assigned to a night shift, \CONFIRM{their preferences should be prioritized in the daytime shift}. Furthermore, in the following day, they should be assigned to an absense, whenever possible.
    }
    \item {
        \label{criterion:worker-groups}
        \NOIMPL{Some workers go to the gym together and, thus, prefer to have similar schedules.}
    }
\end{enumerate}

As it can be seen, this problem presents similarities to the \textit{nurses' problem}, which is a well known problem in the scheduling literature. However, this problem has some unique constraints, such as the fact that some shifts have variable length or that a pharmacist can be assigned to a daytime shift and a nighttime shift in the same day. Furthermore, this problem also presents some optimization elements, such as the equal distribution of overtime and absenses.

\section{Proposed Model}
\label{section:proposed-model}

To approach this problem, a Constraint Logic Programming (CLP) based approach was used. CLP allows for the easy modeling of constraints and the optimization of a given objective function.
The proposed model was implemented using SICStus Prolog 4.7.1. 

The model can be divided into three main parts: the data structures used, the constraints that are imposed on the data structures, and the objective function that is optimized.

\subsection{Data Structures}
\label{section:data-structures}

Regarding the data structures that were used, an approach similar to the one used in the \textit{nurses' problem} was used. That is two matrices were created, using the \texttt{setup\_domain\_and\_channeling/5} predicate:

\begin{itemize}
    \item {
        The \texttt{Day\_Worker\_Shift} matrix, where, for each day, each worker is assigned a shift. If a worker doesn't get assigned to a shift on that day, then the value assigned is 0.
        The domain for these values is \texttt{0..S}, where \texttt{S} is the number of shifts. For each day, the constraint \texttt{all\_distinct\_except\_0} is imposed, so that a shift is not assigned to more than one worker.
    }
    \item {
        The \texttt{Day\_Shift\_Worker} matrix, where, for each day, each shift is assigned a worker. If a shift doesn't get assigned to a worker on that day, then the value assigned is 0.
        The domain for these values is \texttt{0..W}, where \texttt{W} is the number of workers. For each day, the constraint \texttt{all\_distinct\_except\_0} is imposed, so that a worker is not assigned to more than one shift.
    }
\end{itemize}

After having these matrices set up, they need to be channeled, so that changes in one matrix are reflected in the other.
This is done by iterating over the matrices and applying \texttt{element} and \texttt{\#<=>} constraints. Regarding the \texttt{element} constraints, they are used because the values in \texttt{Day\_Worker\_Shift} matrix correspond to indices in the \texttt{Day\_Shift\_Worker} matrix. The same is true for the values in the \texttt{Day\_Shift\_Worker} matrix and the indices in the \texttt{Day\_Worker\_Shift} matrix. The \texttt{\#<=>} constraints are used to ensure the \texttt{element} constraints are only applied when the value used as index is different from 0.

Regarding the night shifts, using what was described above isn't enough, since the approach doesn't allow a worker to be assigned to a day and a night shift on the same day, as described in constraint \ref{constraint:nighttime-shift-implies-daytime-shift}.
To solve this, the night shifts have their own \texttt{Day\_Worker\_Shift} and \texttt{Day\_Shift\_Worker} matrices, named \texttt{Day\_Worker\_NightShift} and \texttt{Day\_NightShift\_Worker}, respectively.

\subsection{Constraints}
\label{section:constraints}

For each constraint that was described in section \ref{section:problem-statement}, a predicate was created to enforce it. How each predicate is implemented will be described in the following paragraphs.

Regarding constraints \ref{constraint:domain} and \ref{constraint:shifts}, they already already enforced by the domain of the matrices, as described in \ref{section:data-structures}.

Regarding the optimization criterion \ref{criterion:worker-groups}, it was not implemented, since it was considered to be a low priority criterion.

\subsubsection*{Variable-length shifts}

In order to implement variable-length shifts, as described in constraint \ref{constraint:variable-length-shifts}, the concept of alternative shifts was introduced. Alternative shifts are shifts where only of them can exist in a given day of the week. A list of pairs of indices of shifts is defined to indicate which shifts represent different lengths of the same shift.
To implement this constraint, the \texttt{setup\_alternative\_shifts/2} predicate was created. In it, for each day in the \texttt{Day\_Shift\_Worker} matrix, two sets of constraints are imposed: the first is a constraint that states that the number of shifts that have no workers assigned to them is equal to the number of alternative shift pairs, implemented using \texttt{count}; the second is that, for each pair of alternative shifts, exactly one of them must have a worker assigned, implemented using \texttt{Worker1 \#= 0 \#<=> Worker2 \#\textbackslash= 0}.

\subsubsection*{Overtime shifts}

To implement overtime shifts, as described in constraint \ref{constraint:overtime-shifts}, a list of indices of shifts that are considered to be overtime shifts is defined. This approach, combined with the one used for variable-length shifts, allows for only one of the alternative shifts to be considered an overtime shift, if needed. Furthermore, the total number of overtime shifts per day is defined.
To implement this constraint, the \texttt{setup\_overtime\_shifts/3} predicate was created. In it, for each day in the \texttt{Day\_Worker\_Shift} matrix, constraints are imposed, using the \texttt{count} constraint, to determine how many times each overtime shift occurs in that day. The counts are then added, using the \texttt{sum} constraint, and the result is imposed to be \CONFIRM{equal} to the daily limit of overtime shifts.

\subsubsection*{Late shifts}

To implement late shifts, as described in constraint \ref{constraint:late-shifts}, a list of indices of shifts that are considered to be late shifts is defined. Like with the overtime shifts, this allows for only one of the alternative shifts to be considered a late shift, if needed. Furthermore, the maximum number of late shifts per week per worker is defined.
To implement this constraint, the \texttt{setup\_late\_shifts/3} predicate was created. In it, for each worker in the \texttt{Worker\_Day\_Shift} matrix (the transposed matrix of the \texttt{Day\_Worker\_Shift} matrix), constraints are imposed, using the \texttt{count} constraint, to determine how many times each late shift occurs in that week. The counts are then added, using the \texttt{sum} constraint, and the result is imposed to be less than or equal to the weekly limit of late shifts per worker.

\subsubsection*{Nighttime shift implies an early daytime shift}

To implement the constraint that assigning a nighttime shift implies an assignment of an daytime shift, as described in constraint \ref{constraint:nighttime-shift-implies-daytime-shift}, the \texttt{setup\_night\_shifts/2} predicate was created. In it, for each day in the \texttt{Day\_Worker\_Shift} and \texttt{Day\_Worker\_NightShift} matrices, constraints are imposed to ensure that, if a worker is assigned to a nighttime shift on a given day, then they must also be assigned to a daytime shift on the same day. This is implemented using the constraint \texttt{NightShift \#\textbackslash= 0 \#=> Shift \#\textbackslash= 0}. To ensure that the assigned daytime shift is an early shift, as described in the previously mentioned constraint, the \texttt{setup\_late\_night\_shifts/3} predicate was implemented. In it, for each day in the \texttt{Day\_Shift\_Worker} and \texttt{Day\_Worker\_NightShift} matrices, all late shifts are iterated over and, if a worker is assigned to a late shift during the day, then their nighttime shift is set to 0, using the \texttt{element} constraint.

\subsubsection*{Shift incompatibilities}

To implement shift incompatibilities, as described in constraint \ref{constraint:shift-incompatibilities}, a list of shift incompatibilities is defined. This list contains pairs of indices of shifts that are incompatible with each other. For each worker, a list of indices of shifts that are incompatible with their daily commitments is defined. This list is then used to impose constraints on the \texttt{Day\_Worker\_Shift} matrix.
To implement this constraint, the \texttt{setup\_shift\_incompatibilities/2} predicate was created. In it, for each day in the \texttt{Day\_Worker\_Shift} matrix, constraints are imposed to ensure that a worker is not assigned to shifts that are incompatible with their daily commitments. This is implemented using the \texttt{element} constraint.

\section{Results}
\label{section:results}

\section{Further Work}
\label{section:further-work}

\section{Conclusion}
\label{section:conclusion}

\color{red}
\section{Ease of Use}

\subsection{Maintaining the Integrity of the Specifications}

The IEEEtran class file is used to format your paper and style the text. All margins, 
column widths, line spaces, and text fonts are prescribed; please do not 
alter them. You may note peculiarities. For example, the head margin
measures proportionately more than is customary. This measurement 
and others are deliberate, using specifications that anticipate your paper 
as one part of the entire proceedings, and not as an independent document. 
Please do not revise any of the current designations.

\section{Prepare Your Paper Before Styling}

Before you begin to format your paper, first write and save the content as a 
separate text file. Complete all content and organizational editing before 
formatting. Please note sections \ref{AA}--\ref{SCM} below for more information on 
proofreading, spelling and grammar.

Keep your text and graphic files separate until after the text has been 
formatted and styled. Do not number text heads---{\LaTeX} will do that 
for you.

\subsection{Abbreviations and Acronyms}\label{AA}

Define abbreviations and acronyms the first time they are used in the text, 
even after they have been defined in the abstract. Abbreviations such as 
IEEE, SI, MKS, CGS, ac, dc, and rms do not have to be defined. Do not use 
abbreviations in the title or heads unless they are unavoidable.

\subsection{Units}

\begin{itemize}
\item Use either SI (MKS) or CGS as primary units. (SI units are encouraged.) English units may be used as secondary units (in parentheses). An exception would be the use of English units as identifiers in trade, such as ``3.5-inch disk drive''.
\item Avoid combining SI and CGS units, such as current in amperes and magnetic field in oersteds. This often leads to confusion because equations do not balance dimensionally. If you must use mixed units, clearly state the units for each quantity that you use in an equation.
\item Do not mix complete spellings and abbreviations of units: ``Wb/m\textsuperscript{2}'' or ``webers per square meter'', not ``webers/m\textsuperscript{2}''. Spell out units when they appear in text: ``. . . a few henries'', not ``. . . a few H''.
\item Use a zero before decimal points: ``0.25'', not ``.25''. Use ``cm\textsuperscript{3}'', not ``cc''.)
\end{itemize}

\subsection{Equations}

Number equations consecutively. To make your 
equations more compact, you may use the solidus (~/~), the exp function, or 
appropriate exponents. Italicize Roman symbols for quantities and variables, 
but not Greek symbols. Use a long dash rather than a hyphen for a minus 
sign. Punctuate equations with commas or periods when they are part of a 
sentence, as in:
\begin{equation}
a+b=\gamma\label{eq}
\end{equation}

Be sure that the 
symbols in your equation have been defined before or immediately following 
the equation. Use ``\eqref{eq}'', not ``Eq.~\eqref{eq}'' or ``equation \eqref{eq}'', except at 
the beginning of a sentence: ``Equation \eqref{eq} is . . .''

\subsection{\LaTeX-Specific Advice}

Please use ``soft'' (e.g., \verb|\eqref{Eq}|) cross references instead
of ``hard'' references (e.g., \verb|(1)|). That will make it possible
to combine sections, add equations, or change the order of figures or
citations without having to go through the file line by line.

Please don't use the \verb|{eqnarray}| equation environment. Use
\verb|{align}| or \verb|{IEEEeqnarray}| instead. The \verb|{eqnarray}|
environment leaves unsightly spaces around relation symbols.

Please note that the \verb|{subequations}| environment in {\LaTeX}
will increment the main equation counter even when there are no
equation numbers displayed. If you forget that, you might write an
article in which the equation numbers skip from (17) to (20), causing
the copy editors to wonder if you've discovered a new method of
counting.

{\BibTeX} does not work by magic. It doesn't get the bibliographic
data from thin air but from .bib files. If you use {\BibTeX} to produce a
bibliography you must send the .bib files. 

{\LaTeX} can't read your mind. If you assign the same label to a
subsubsection and a table, you might find that Table I has been cross
referenced as Table IV-B3. 

{\LaTeX} does not have precognitive abilities. If you put a
\verb|\label| command before the command that updates the counter it's
supposed to be using, the label will pick up the last counter to be
cross referenced instead. In particular, a \verb|\label| command
should not go before the caption of a figure or a table.

Do not use \verb|\nonumber| inside the \verb|{array}| environment. It
will not stop equation numbers inside \verb|{array}| (there won't be
any anyway) and it might stop a wanted equation number in the
surrounding equation.

\subsection{Some Common Mistakes}\label{SCM}
\begin{itemize}
\item The word ``data'' is plural, not singular.
\item The subscript for the permeability of vacuum $\mu_{0}$, and other common scientific constants, is zero with subscript formatting, not a lowercase letter ``o''.
\item In American English, commas, semicolons, periods, question and exclamation marks are located within quotation marks only when a complete thought or name is cited, such as a title or full quotation. When quotation marks are used, instead of a bold or italic typeface, to highlight a word or phrase, punctuation should appear outside of the quotation marks. A parenthetical phrase or statement at the end of a sentence is punctuated outside of the closing parenthesis (like this). (A parenthetical sentence is punctuated within the parentheses.)
\item A graph within a graph is an ``inset'', not an ``insert''. The word alternatively is preferred to the word ``alternately'' (unless you really mean something that alternates).
\item Do not use the word ``essentially'' to mean ``approximately'' or ``effectively''.
\item In your paper title, if the words ``that uses'' can accurately replace the word ``using'', capitalize the ``u''; if not, keep using lower-cased.
\item Be aware of the different meanings of the homophones ``affect'' and ``effect'', ``complement'' and ``compliment'', ``discreet'' and ``discrete'', ``principal'' and ``principle''.
\item Do not confuse ``imply'' and ``infer''.
\item The prefix ``non'' is not a word; it should be joined to the word it modifies, usually without a hyphen.
\item There is no period after the ``et'' in the Latin abbreviation ``et al.''.
\item The abbreviation ``i.e.'' means ``that is'', and the abbreviation ``e.g.'' means ``for example''.
\end{itemize}
An excellent style manual for science writers is \cite{report:zhu2024scalable}.

\subsection{Authors and Affiliations}
\textbf{The class file is designed for, but not limited to, six authors.} A 
minimum of one author is required for all conference articles. Author names 
should be listed starting from left to right and then moving down to the 
next line. This is the author sequence that will be used in future citations 
and by indexing services. Names should not be listed in columns nor group by 
affiliation. Please keep your affiliations as succinct as possible (for 
example, do not differentiate among departments of the same organization).

\subsection{Identify the Headings}
Headings, or heads, are organizational devices that guide the reader through 
your paper. There are two types: component heads and text heads.

Component heads identify the different components of your paper and are not 
topically subordinate to each other. Examples include Acknowledgments and 
References and, for these, the correct style to use is ``Heading 5''. Use 
``figure caption'' for your Figure captions, and ``table head'' for your 
table title. Run-in heads, such as ``Abstract'', will require you to apply a 
style (in this case, italic) in addition to the style provided by the drop 
down menu to differentiate the head from the text.

Text heads organize the topics on a relational, hierarchical basis. For 
example, the paper title is the primary text head because all subsequent 
material relates and elaborates on this one topic. If there are two or more 
sub-topics, the next level head (uppercase Roman numerals) should be used 
and, conversely, if there are not at least two sub-topics, then no subheads 
should be introduced.

\subsection{Figures and Tables}
\paragraph{Positioning Figures and Tables} Place figures and tables at the top and 
bottom of columns. Avoid placing them in the middle of columns. Large 
figures and tables may span across both columns. Figure captions should be 
below the figures; table heads should appear above the tables. Insert 
figures and tables after they are cited in the text. Use the abbreviation 
``Fig.~\ref{fig}'', even at the beginning of a sentence.

\begin{table}[htbp]
\caption{Table Type Styles}
\begin{center}
\begin{tabular}{|c|c|c|c|}
\hline
\textbf{Table}&\multicolumn{3}{|c|}{\textbf{Table Column Head}} \\
\cline{2-4} 
\textbf{Head} & \textbf{\textit{Table column subhead}}& \textbf{\textit{Subhead}}& \textbf{\textit{Subhead}} \\
\hline
copy& More table copy$^{\mathrm{a}}$& &  \\
\hline
\multicolumn{4}{l}{$^{\mathrm{a}}$Sample of a Table footnote.}
\end{tabular}
\label{tab1}
\end{center}
\end{table}

\begin{figure}[htbp]
\centerline{\includegraphics{assets/fig1.png}}
\caption{Example of a figure caption.}
\label{fig}
\end{figure}

Figure Labels: Use 8 point Times New Roman for Figure labels. Use words 
rather than symbols or abbreviations when writing Figure axis labels to 
avoid confusing the reader. As an example, write the quantity 
``Magnetization'', or ``Magnetization, M'', not just ``M''. If including 
units in the label, present them within parentheses. Do not label axes only 
with units. In the example, write ``Magnetization (A/m)'' or ``Magnetization 
\{A[m(1)]\}'', not just ``A/m''. Do not label axes with a ratio of 
quantities and units. For example, write ``Temperature (K)'', not 
``Temperature/K''.

\section*{Acknowledgment}

The preferred spelling of the word ``acknowledgment'' in America is without 
an ``e'' after the ``g''. Avoid the stilted expression ``one of us (R. B. 
G.) thanks $\ldots$''. Instead, try ``R. B. G. thanks$\ldots$''. Put sponsor 
acknowledgments in the unnumbered footnote on the first page.

\section*{References}

Please number citations consecutively within brackets \cite{report:zhu2024scalable}. The 
sentence punctuation follows the bracket \cite{report:zhu2024scalable}. Refer simply to the reference 
number, as in \cite{report:zhu2024scalable}---do not use ``Ref. \cite{report:zhu2024scalable}'' or ``reference \cite{report:zhu2024scalable}'' except at 
the beginning of a sentence: ``Reference \cite{report:zhu2024scalable} was the first $\ldots$''

Number footnotes separately in superscripts. Place the actual footnote at 
the bottom of the column in which it was cited. Do not put footnotes in the 
abstract or reference list. Use letters for table footnotes.

Unless there are six authors or more give all authors' names; do not use 
``et al.''. Papers that have not been published, even if they have been 
submitted for publication, should be cited as ``unpublished'' \cite{report:zhu2024scalable}. Papers 
that have been accepted for publication should be cited as ``in press'' \cite{report:zhu2024scalable}. 
Capitalize only the first word in a paper title, except for proper nouns and 
element symbols.

For papers published in translation journals, please give the English 
citation first, followed by the original foreign-language citation \cite{report:zhu2024scalable}.

And to test, we'll do \cite{IEEEexample:bluebookstandard}. We can also cite our own references, using \cite{report:zhu2024scalable}.

\color{black}
\bibliographystyle{./IEEEtran}
\bibliography{./IEEEabrv,./IEEEexample,./report}

\vspace{12pt}

\color{red}
IEEE conference templates contain guidance text for composing and formatting conference papers. Please ensure that all template text is removed from your conference paper prior to submission to the conference. Failure to remove the template text from your paper may result in your paper not being published.


\end{document}
